\section{Evaluation of the system in general}

\subsection{NUI guidelines}

Direct manipulation techniques: always inline manipulation, no external prompts (except virtual keyboard on the iPad if no external keyboard is connected (which it hopefully is). Deletion option (with confirmation) on each issue.

\subsection{10 meaningful screens}

Haha, 10. Each card state, hint, everything

% TODO: Hint to more content (arrow or something at the end of swimlane)

\begin{itemize}
  \item Swimlanes
  \item Single issues
  \item Issue that has been modified
  \item Issue that is marked for deletion (red: \textbf{occlusion}) 
  \item Issue that is being moved (hint)
  \item Issue that is being moved (escape sequence)
  \item Issue that is being edited (title)
\end{itemize}

\subsection{Five actionable events}

Card deleted (realtime updates of the board to see what is going on?), but mostly "the time is X and we have Y tasks left to complete", or "we have way too many tasks in progress, please fix" (this falls under valuable outcome)

Potential: see who is assigned to the issue

Potential: see how long tasks are estimated to take, and prioritise based on that.

\subsection{Valueable outcomes}

\subsubsection{Short-term valuable outcome}

Immediately see and update the state of the project

Extremely simple interaction optimised for doing when travelling, can be used for walking, immediate overview of the entire project

\subsubsection{Long-term valuable outcome}

Be able to keep track of projects over time (but not across projects), be able to keep a continuous flow of issues % TODO: Talk about archive possibility, statistics

Focus on touch, more tactile approach, bring issues closer by letting people actually interact with them (tangible user interface?) – physical form to digital information (which is really the basis of NUI in the first place, no? – but not really, as there are no real objects; only digital visual representations of the actual work)
