\section{Evaluation of the system in general}

\subsection{NUI guidelines}

Direct manipulation techniques: always inline manipulation, no external prompts (except virtual keyboard on the iPad if no external keyboard is connected (which it hopefully is). Deletion option (with confirmation) on each issue.

Spatial 2D NUI guidelines: environment is optimised for touch through large elements (15mm in all directions, at least 5mm between elements on tiny tablet screen, all touch targets are equally sized),

does \textbf{not} allow users to change the layout because it is not needed, only one screen -> simple to position controls consistently (but consistent across swimlanes),

spatial relationships: higher up means higher rank/priority/importance (it is a hierarchy), %todo: While not explicitly expressed, one may assume that the user's model of the system identifies issues higher up (with a lower index) as more important. After all, the grid layout imposes a hierachical style on the application. This interpretation is, however, left to the user.

todo, doing, done are obviously positioned in that order (we read and work from left to right in the West),

Consistent design with only one page,

not a multi-user system (although it could have been with multi-touch recognition, which is not supported by ng-sortable but could be interesting, would not let more than one user change everyone's view, hard to indicate ownership, only uses flat/wide "navigation" for controls, no hierarchies (e.g., dropdowns),

main view has only important controls and objects, but is not too sparse

% TODO: remove this separator between NUI guidelines and commonsensical/practical stuff

Touch input: Fat fingers: no buttons (only large areas), large margins, everything can be dragged (no specific drag handles), may cause trouble with scrolling

Sources of errors: Fat Fingers, multiple capture states, accidental activation (arm brush – doing nothing to prevent this as of yet), physical manipulation constraints, stolen capture, tabletop debris (it's an iPad – can't really get around that one)

Inter-user task coupling (irrelevant because no two users will use the system at the same time – but could implement this with sockets so that both users will always know what the current status is)


All objects on screen (except \textit{swimlanes}) are interactive (editable, draggable, or sortable) through touch. Landing a finger on a card allows the user to drag the card to another position by sliding onto the closest new position and release the card to the new position by lifting the finger off the card. Landing a finger specifically on an editable text area on the card (the title) and immediately lifting the finger off, then editing the text with help of a keyboard. In this case a \textit{gesture} \parencite[157]{wigdow-wixon:brave-nui-world:2011} as an exit sequence would be helpful, but this is not implemented in the prototype.

% TODO: Reference here
INRC: \textbf{Identity} (move issue or start changing name always does just that), \textbf{Negation} (move back, change back always undoes – could look into gestures like slide to delete, slide back to restore item which is already a thing), \textbf{Reciprocal} (no undo button or anything like it, but could easily implement one), Commutative (changing name, then moving = moving, then changing name) % TODO: Look into and discuss possibilities for gestures and undo button

\subsection{10 meaningful screens}

Haha, 10. Each card state, hint, everything

% TODO: Hint to more content (arrow or something at the end of swimlane)

\begin{itemize}
  \item Swimlanes
  \item Single issues
  \item Issue that has been modified
  \item Issue that is marked for deletion (red: \textbf{occlusion}) 
  \item Issue that is being moved (hint)
  \item Issue that is being moved (escape sequence)
  \item Issue that is being edited (title)
\end{itemize}

\subsection{Five actionable events}

Card deleted (realtime updates of the board to see what is going on?), but mostly "the time is X and we have Y tasks left to complete", or "we have way too many tasks in progress, please fix" (this falls under valuable outcome)

Potential: see who is assigned to the issue

Potential: see how long tasks are estimated to take, and prioritise based on that.

Potential for different input:
\begin{itemize}
  \item Not using a mouse (but there's nothing in the way of using the app in a browser)
  \item Not using a stylus (although recognising a stylus and allowing in-place drawing of "attachments" for issues would be interesting to look into)
\end{itemize}

\subsection{Valueable outcomes}

\subsubsection{Short-term valuable outcome}

Immediately see and update the state of the project

Extremely simple interaction optimised for doing when travelling, can be used for walking, immediate overview of the entire project

\subsubsection{Long-term valuable outcome}

Be able to keep track of projects over time (but not across projects), be able to keep a continuous flow of issues % TODO: Talk about archive possibility, statistics

Focus on touch, more tactile approach, bring issues closer by letting people actually interact with them (tangible user interface?) – physical form to digital information (which is really the basis of NUI in the first place, no? – but not really, as there are no real objects; only digital visual representations of the actual work)
