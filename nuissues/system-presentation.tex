\section{NUIssues}

\subsection{Rationale behind building an issue tracker}

The system is single-user, so inter-user task coupling has not been explicitly addressed, but it is possible to evolve the system to other forms of interfaces like touch walls or touch tables (or even just real-time updates for a project).

Inter-user task coupling (irrelevant because no two users will use the system at the same time – but could implement this with sockets so that both users will always know what the current status is)
  % TODO: Look into socket support for multi-user updates, in which case we probably move to lightly coupled tasks
  % TODO: figure out/discuss what type of coupling the tasks of updating an issue tracker are

\subsection{Scope}

Type: NUI, (not really) TUI, no VR/AR/MR. Obviously not a CLI because it has a GUI.

Form factor: designed only for iPad 1/2/Mini, but can easily write responsive CSS to support more types of devices (relevant for touch tables, smart boards, touch walls, smartphones). Hopefully keyboard is connected. Designed to be used when sitting down, although can be just opened (not interacted with) to grok the project's state on the move. Designed to be interacted with only when necessary and looked at when information is required

\subsection{Sketches}

% TODO: hack something together in Balsamiq and/or Photoshop to show that this was well planned out

\subsection{Development process}

Web client can be used by all types of devices (phones, tablets, touch tables, "smartboards", touch walls)

No specific SDKs have been used, although several frameworks are used:
\begin{itemize}
  \item AngularJS (front-end) with ng-sortable, bootstrap 3, custom CSS
  \item Node.js (back-end) with several components (Express web server, Mongoose ODM (could just as easily have used relations), body-parser)
  \item MongoDB document database (could just as easily have used relations for this simple use case)
  \item Potential: JWT (or just cookies): save data inside the token
\end{itemize}

Could have built something native for better performance and complete control of the environment, but takes far too much time and was beside the concept we set out to prove.
