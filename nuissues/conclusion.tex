\section{Conclusion and future work}

Focusing on touch provides a more tactile approach to issue tracking, which brings issues closer by letting people actually interact with them. Giving a physical representation that can be naturally interacted with is, after all, the basis of NUI design, and this will both optimise workflows where touch is a better interface type than a mouse (for example when manipulating a project on the go).

Following best practises and developing new guidelines that can eventually be commonly agreed upon is the only way of moving NUI forward. We have considered a few of Microsoft's attempts at NUI design, and also considered implementations like Apple's Touch ID and the ethical controversy related to that system.

NUIssues is a single-user system, and it would be interesting to explore multi-user issue trackers where the system for example presents several projects at the same time, and allows different teams to visually plan progress and coordinate work.

The \textit{cards} in NUIssues are conventional, rectangular cards. An interesting future project can look into the generation of iceberg targets \autocite[91]{wigdow-wixon:brave-nui-world:2011}.

Optimising an application for touch and focusing on content, while simultaneously focusing on business value is the best and only way of designing a system to be as efficient as possible for its purpose. Naturally, introducing new interface types makes it more difficult to stay true to a single interface type, and will be expensive, but is in some cases worth it.

For an issue tracker, updating issues on the go is a new option enabled by NUIssues. Even though some services (like Trello) already have similar functionality on mobile, they have a distinct lack of certain functions such as worklogs, which are critical in some businesses like consultancy companies. Optimising for touch brings the tasks closer, and opens up a world of new opportunities. While 2010 may not be the year we look back to and remember as the year we left old-fashioned keyboard-and-mouse input behind, we are certainly in the \textit{era} that will be remembered for it.
