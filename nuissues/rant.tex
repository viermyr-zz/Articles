\section{Rant}

\subsection{Practical stuff}

Link to application: % TODO: add link to app (Heroku or VPS?)

\subsection{Knowledge}

NUI in software integration (obviously integrates with a different system, and provides valuable information about the current state of the company's project(s) and thus integrates with the company as well. Very easy to set up (many companies already have iPads hanging around – just get some more, or open a browser).

Motion controllers, gaze trackers, and brain trackers may be less interesting to use here (although Atlassian has been able to use brain power to move issues between swimlanes at a hackathon – or so they say.



Important theories:

\begin{itemize}
  \item Type of interface: somewhere in between a GUI and a NUI (the system has a GUI which is "optimised" for the natural interface touch, built for tablets with an existing touch screen. CLI only used for development :)
  \item Behaviour models: predictive models
  \item Hick-Hyman Law: about reaction time when presented with a bunch of options, not really relevant as there are zero menus (for which the law is mostly being used)
  \item Keystroke-level model (KLM): Key stroking only relevant when adding or editing issues, pointing relevant because the finger moves, homing time minimised because of large, responsive controls, drawing minimised (and destinations hinted), mental operator, system response operator (hello Heroku) % TODO look into mental operator and System Response operator
  \item Motor behaviour models: descriptive models: (state 0: waiting for stimuli) -> <finger down on issue handle (entire issue)> -> ((state 1: dragging issue) -> <finger up above legal area> -> (state 2: releasing "dropping" issue) -> (state 0: waiting for stimuli)) OR (<finger up from text> -> (state 4: editing text) -> <tap "save changes" control> -> (state 0: waiting for stimuli))
  \item Mental models: HCI Design: target system (duh), conceptual model (issue tracker with touch interface which supports only adding and moving/reordering issues), system image (the materialised version of the designer's mental model of the system), user's mental model of the system image (hopefully very close, but discuss this), designer's user model (hopefully very close, draw contrasts and use same metaphors/stuff as Trello, JIRA, Asana, ding.io, ServiceNow, Symfoni Notes, other issue trackers users have already used, and simple post-it note setups)
  \item Spatial 2D NUI guidelines: environment is optimised for touch through large elements (15mm in all directions, at least 5mm between elements on tiny tablet screen, all touch targets are equally sized), does \textbf{not} allow users to change the layout because it is not needed, only one screen -> simple to position controls consistently (but consistent across swimlanes), spatial relationships: higher up means higher rank/priority/importance (it is a hierarchy), todo, doing, done are obviously positioned in that order, naturally consistent, not a multi-user system (although it could have been with multi-touch recognition, which is not supported by ng-sortable but could be interesting, would not let more than one user change everyone's view, hard to indicate ownership, only uses flat/wide "navigation" for controls, no hierarchies (e.g., dropdowns), main view has only important controls and objects, but is not too sparse
  \item Touch input: Fat fingers: no buttons (only large areas), large margins, everything can be dragged (no specific drag handles), may cause trouble with scrolling
  \item Could be very cool to look at iceberg target implementation in a swimlane application % TODO: define "swimlane" and "issue"
  \item All objects on screen (except swimlanes) are draggable/sortable through land on (start drag) and lift off (release to new position) except when that means "edit text inline", but sliding onto a swimlane's issue list when dragging an issue causes a hint for the next free position, and sliding off the entire board when dragging an issue results in the issue not being moved % TODO: Does it result in the object (issue) not being moved?)
  \item Sources of errors: Fat Fingers, multiple capture states, accidental activation (arm brush – doing nothing to prevent this as of yet), physical manipulation constraints, stolen capture, tabletop debris (it's an iPad – can't really get around that one)
  \item Gamification: no users (and no points)\dots % TODO: look into and discuss how gamification could be utilised in a task tracker (public scoreboards, points, badges)
  \item Not using a mouse
  \item Not using a stylus (although recognising a stylus and allowing in-place drawing of "attachments" for issues would be interesting to look into)
  \item NUI gestures: not really used (should give escape event)
  \item INRC: \textbf{Identity} (move issue or start changing name always does just that), \textbf{Negation} (move back, change back always undoes – could look into gestures like slide to delete, slide back to restore item which is already a thing), \textbf{Reciprocal} (no undo button or anything like it, but could easily implement one), Commutative (changing name, then moving = moving, then changing name) % TODO: Look into and discuss possibilities for gestures and undo button
\end{itemize}

Ethical problems: none in this application (except for security, duh), but certainly applicable % TODO: Digress into short discussion of ethical issues generally raised in NUI/Interactive Technologies contexts

Short literature review % TODO: Read some papers, especially Ravi's

Conclusion: Stuff has been done, stuff is lacking, more stuff needs to be done in stuff areas