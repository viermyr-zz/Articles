\section{Introduction}

The February \citeyear{hbr:incompleteleader} issue of \emph{Harvard Business Review} features the article \citetitle{hbr:incompleteleader}, which argues that it is time to end the myth of the complete, flawless leader at the top \autocite{hbr:incompleteleader}. Instead, for the best of organisations, the article introduces the concept of the \emph{incomplete} (but not incompetent) leader, who knows when to let the experts take action in their own fields.

\citetitle{agilemanifesto} (\citeauthor{agilemanifesto}) was signed by some of the leading names in software development practises, and published in \citeyear{agilemanifesto}.While focused specifically on the software development industry, its concern for individuals and interactions over processes and tools touches self-organising teams – which are at the heart of the theory of the incomplete leader.

\citeauthor{fernandez:agilevstraditional} (\citeyear{fernandez:agilevstraditional}) argue that over the last few decades, in the center of increased globalisation, corporations have been changing their project management from hierarchical to collaborative. This fits the sentiments of \citetitle{hbr:incompleteleader} and \citetitle{agilemanifesto}, and is especially clear in the software industry.

With the changes to project structure and management in combination with increased globalisation, the \emph{role} of the project manager has naturally had to be adapted. Once an all-knowing role with a firm grip on the project in its entirety, the project manager (or leader) is now be in charge of ensuring that a self-organising team works efficiently and communicates well.

Several methodologies and frameworks have emerged to counter the insecurity in projects, and many of these build on the \emph{agile} principles. Most of the literature on agile relates to the IT industry \autocite{fernandez:agilevstraditional}, and this article is no exception. However, much of the content should be applicable to other industries as well.

This article first considers and provides a brief review of the existing research on the role of the project leader in agile projects. Next, it reflects upon the state of the research on this area. Following this reflection, it discusses the findings. In conclusion, it summarises the key points regarding the role of the project leader in agile projects.
