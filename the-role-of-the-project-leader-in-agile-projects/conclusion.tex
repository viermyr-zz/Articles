\section{Conclusion}

The research body on agile project management is becoming relatively large, although a significant portion of it is still in the form of case studies. This is an issue the academic community should aim to resolve, for instance by aggregating these case studies into statistical data.

We have seen that there are many aspects to the role as a project leader in an agile project, and that the agile project manager's job differs from that of the traditional project manager. In traditional project management, the project manager is expected to follow a timeline and minimise time and money spent on the project while still delivering a predefined product on time. In agile projects, business value is in focus and the final product may not be defined at all.

There are several strategies and tools to choose from in an agile project, and the project leader must be familiar with the key differences between different strategies and tools in order to be able to pick the most suitable set of tools for the project at hand; agile may not always be the correct approach, in which case the project leader may need to take on the role as traditional project manager. A good project leader cannot be familiar with only one small set of tools.

Ensuring that the team works in an agile fashion and truly understands how the implementation of the agile principles positively affect the development. Only in this way can the team function in a truly agile fashion: it is not enough to employ a set of rules. This may be extra important when introducing agile for the first time in a well-established business.

Agile projects are about business value, and proper communication with the product owner is key. It is the project leader's responsibility that any impediments with this communication channel are removed, for the best of the entire team.

Other factors such as project scale and distributed teams may also affect the ability to utilise agile principles to improve the development process.
