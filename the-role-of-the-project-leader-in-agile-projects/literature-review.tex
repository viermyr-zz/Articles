\section{A brief literature review}

\citeauthor{fernandez:agilevstraditional} (\citeyear{fernandez:agilevstraditional}) identified the literature on agile project management as still in its infancy. Especially, they emphasise the importance of advancing knowledge in scopes outside software development. While the statement still holds true, the Project Management Body of Knowledge (PMBOK) now has a growing category named \emph{Agile} \parencite{pmbok} with many software-unrelated entries.

\citeauthor{sutling:taxonomy} (\citeyear{sutling:taxonomy}) have already conducted a fairly extensive literature review on important project leader \emph{behaviour} in agile projects. In their attempt to further this knowledge with focus on projects conducted in Malaysia, they have identified tens of publications from other regions. It becomes clear in this article, however, that a relatively large research body exists on the topic of modern project management and how agile teams interact – especially in the software industry.

From their paper, however, it seems that a very large portion of the research body consists of \emph{case studies} rather than statistics. Thus, the academic focus could benefit from aggregating more data for statistical analysis.

What also becomes clear is that there is great focus on self-organisation and self-managing teams. This may be due to the fact that many of the agile methodologies, frameworks, or strategies do not actually involve a \emph{project manager} in the traditional sense; instead, a \emph{coach} role is defined to ensure that everyone follows the agile principles. In the very widespread methodology \emph{Scrum}, for instance, the \emph{ScrumMaster} role's main responsibility is ensuring that the team properly follows Scrum \autocite{schwaber:scrum}. The role's few other responsibilities include leading meetings where necessary, and removing impediments and shielding the development team from stakeholder involvement during its time-boxed iterations.

As a result, there is little literature on how the traditional project manager adapts to the agile principles and mindset – there is no call for a commanding and controlling role, so a good traditional project \emph{manager} may not be the best fit for the role as an agile project \emph{leader}. One article concerned with this matter is \citeauthor{yi:managerasscrummaster}'s \citetitle{yi:managerasscrummaster} (\citeyear{yi:managerasscrummaster}), which is an experience report.
