\section{Literature Review: State of the Art}

\subsection{\citetitle{xu-he-li:internet-of-things-in-industries-a-survery:2014} (\citeyear{xu-he-li:internet-of-things-in-industries-a-survery:2014})}

\textcite{xu-he-li:internet-of-things-in-industries-a-survery:2014} contributes a major review of the current research on the Internet of Things (IoT). This very recent (\citeyear{xu-he-li:internet-of-things-in-industries-a-survery:2014}) survey paper with more than than 80 trustworthy references identifies several key gaps in the current knowledge body regarding the Internet of Things.

Additionally, they propose a service-oriented architectural (SOA) style approach to the Internet of Things. This approach is not considered by this article.

\subsection{\citetitle{paganelli-turchi-guili:a-web-of-things-framework-for-restful-applications-and-its-experimentation-in-a-smart-city} (\citeyear{paganelli-turchi-guili:a-web-of-things-framework-for-restful-applications-and-its-experimentation-in-a-smart-city})}

As mentioned in the introduction, \textcite{paganelli-turchi-guili:a-web-of-things-framework-for-restful-applications-and-its-experimentation-in-a-smart-city} describes a lack of actually modelled and implemented applications as a key hole in the current research body.



These publications all seem to agree on one thing: the Internet of Things is still in its infancy, and one of the largest gaps stems from very few actual full-stack implementations on top of the Web (and, by extent, the Internet) of things. We absolutely need to standardise protocols in a good way, and we certainly need to let the Internet and Web of Things evolve together with the actual Web.
