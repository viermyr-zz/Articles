\section{Literature Review: State of the Art}

\textcite{xu-he-li:internet-of-things-in-industries-a-survey:2014} contributes a major review of the current research on the Internet of Things (IoT). This very recent (\citeyear{xu-he-li:internet-of-things-in-industries-a-survey:2014}) survey paper with more than than 80 trustworthy references identifies several key gaps in the current knowledge body regarding the Internet of Things. The main points –- cost, security, standardisation, and technology -- are all areas that will need to be explored further, but only standardisation and technology are considered in this article. Additionally, they propose a service-oriented architecture (SOA) style approach to the Web of Things. This approach is not considered by this article.

As mentioned in the introduction, \textcite{paganelli-turchi-guili:a-web-of-things-framework-for-restful-applications-and-its-experimentation-in-a-smart-city:2014} describes a lack of actually modelled and implemented applications as a key hole in the current research body. This article also refers to a relatively large number of other articles proposing middleware and frameworks for designing applications in the Web of Things.

However, \textcite{palattella-accettura-vilajosana-watteyne-gieco-boggia-dohler:standardized-protocol-stack-for-the-internet-of-important-things:2012} claims that what may have previously seemed impossible given the restrictions of the Internet of Things in terms of building a standards-compliant stack may indeed become a reality. They propose a highly technical communication stack for an entire application, but have not consider actually implementing a system. It is worth noting that their stack includes IETF's RFC 7252 –- the Constrained Application Protocol (COAP) (\citeyear{ietf:the-constrained-application-protocol:2014}) for application layer communication.

\textcite{xu-he-li:internet-of-things-in-industries-a-survey:2014} also mention context awareness as an important factor in the Internet of Things, as millions and billions of sensors will be connected, collectively producing extreme amounts of data. While not considered by this article, using context awareness and artificial intelligence to filter out meaningful, important data will be a great tool as we begin to find more and more use cases for the Internet and Web of Things.

It seems that there is no lack of proposed frameworks, protocols, and standards for connecting things to the internet and making them part of the web. There is no shortage of frameworks for the actual communication between devices and servers, either, and we have quite a few contributions regarding storage of very large numbers of data points. We also have much research on analysing the data on the field of Big Data, but that is outside the scope of this article).

Disregarding cost, privacy, and security, the main problem of the current Web of Things research body seems to arise only when committing to building a complete full-stack application: there is no standard, proven, manufacturer-independent way to implement a complete application for gathering and analysing data from a custom Internet of Things system. Indeed, as \textcite{xu-he-li:internet-of-things-in-industries-a-survey:2014} put it: the Internet of Things is still in its infancy.
