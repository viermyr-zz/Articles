\section{Conclusion}
We have seen that the current body of research on the Internet and Web of Things agrees that standardisation, full-stack research-oriented implementations, technology, and security are among the most important areas to look into in the future. Data Analysis as a Service attempts to address the first two of these issues, and is a small step on the way to bridging the gap. More focus must be directed at full-stack implementations of Internet and Web of Things-oriented applicatons, with special regard to separate use cases and domains. In particular, it should be interesting to see what matters in development of commercial products. 

Utilising existing Web standards instead of developing the Internet and Web of Things as its own technology is going to be an important part of the process of simplifying the Internet of Things. We will probably require some new protocols as well -- CoAP is a great example of this -- but developing the WoT with the Web in mind will be crucial. At present, business needs and proposed technology, frameworks, and protocols are in conflict -- but as more example implementations become available, this will hopefully change. Standardising protocols instead of having manufacturers implement custom means of communication is key to simplifying the Internet and Web of Things.

Security, privacy, cost, and maintenance of a distributed network such as the Internet of Things are still major considerations to make, and are certainly directions in which the academic community should go in the near future.
