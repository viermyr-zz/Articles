\section{Introduction and background}
Starting with \nocite{weiser:the-computer-for-the-21st-century:1991}Mark Weiser's "The Computer for the 21st Century" in 1991, we have seen an enormous development in computer science towards ubiquitous computing and interconnected physical devices -- things.

The Internet of Things (IoT) is perhaps the fastest emerging technology trend of the present time. The IoT technologies and applications are still in their infancy \autocite{xu-he-li:internet-of-things-in-industries-a-survey:2014}, and so the academic community must thoroughly address the area. Although "IoT" was initially meant to describe a network of Radio-Frequency ID-enabled devices, it has since been expanded to the following:

\blockquote[{Kranenburg, 2007 cited in \citeauthor{xu-he-li:internet-of-things-in-industries-a-survey:2014}, \citeyear[1]{xu-he-li:internet-of-things-in-industries-a-survey:2014}}]{a dynamic global network infrastructure with selfconfiguring capabilities based on standard and interoperable communication protocols where physical and virtual ‘Things’ have identities, physical attributes, and virtual personalities and use intelligent interfaces, and are seamlessly integrated into the information network}.

It becomes clear that the Internet of Things indeed encompasses all devices with a sensor, but there is also a second implication: the humongous number of data points that will inevitably be collected is of no use to anyone unless it is processed. The definition also presents us with several implicit challenges, backed by \textcite{xu-he-li:internet-of-things-in-industries-a-survey:2014} and \textcite{palattella-accettura-vilajosana-watteyne-gieco-boggia-dohler:standardized-protocol-stack-for-the-internet-of-important-things:2012}. These include, but are not limited to, privacy, distribution and maintenance, and security concerns in the distributed system that is the IoT. These are all important areas to explore, but outside the scope of this article.

Also important to mention is the Web of Things (WoT): the software layer on top of the Internet of Things. This article mainly focuses on the programming model side of an IoT application, and is thus mostly concerned with the WoT.

Furthermore, \textit{standards} are a real concern. This is described in \textcite{palattella-accettura-vilajosana-watteyne-gieco-boggia-dohler:standardized-protocol-stack-for-the-internet-of-important-things:2012}, which emphasises emerging industry alliances and IEEE/IETF working groups as the key to success.

Finally, the pre-eminent concern of this article is the gap of knowledge with regard to modelling and implementing complete IoT-oriented applications, as described by \textcite{paganelli-turchi-guili:a-web-of-things-framework-for-restful-applications-and-its-experimentation-in-a-smart-city:2014}.

This article first revisits the current state of research on the fields of the Internet and Web of Things, respectively. It then presents an architectural model and proof-of-concept implementation of a full-stack IoT-oriented application which accepts, stores, and provides access to the data in addition to subscription to real-time feeds for new data points. Third, it compares the experiences from modelling and developing the application to the existing research. Lastly, the most important lessons are highlighted and briefly discussed.
