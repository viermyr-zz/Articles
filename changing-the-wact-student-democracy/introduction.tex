\section{Introduction}

The three University Colleges (UCs) Westerdals, the Norwegian School of Information Technology (NITH), and the Nordic Institute of Stage and Studio (NISS) are subject to a merger forced upon them by their common owner. The new UC, Westerdals Oslo ACT, will consist of five different faculties, with varying numbers of students and study programmes.

As a natural consequence, the three UCs's student democracies (SDs) need to be merged in a fashion that accounts for inclusion of all faculties and the distribution of students, maintains the practices that already work well for the existing SDs, and establishes a cross-faculty culture despite the faculties not being colocated in the first years after the merger.

This introduces several changes of not only practical nature, but also cultural, which are often the most difficult to deal with \autocite[12495]{luftman:managing-the-information-technology-resource:2009}.

This article first provides the background for the change effort, before taking a cursory look at important literature in the fields of \textit{change management} and \textit{merging organisations like the student democracy} based on a literature search, with special regard to \textit{ethical issues} one must expect to face in the opposition against the change effort.

Then, using \autocite{kotter:leading-change-why-transformation-efforts-fail:1995} as a framework, it demonstrates a practical plan, as suggested in \autocite{pettigrew-woodman-cameron:studying-organizational-change-and-development:2001}, for merging the three student democracies and how the opposition against the change effort will be handled.

Finally, the article concludes by contextualising this particular merger, points out the key ways of handling the opposition in this particular case, and suggests a direction for further research on the field of handling opposition in a change effort involving mergers.
