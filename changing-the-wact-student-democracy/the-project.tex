\section{Key expected issues for the change effort}

The change effort of merging the SDs has a few different stakeholders:

\begin{itemize}
  \item Of course, the students make up the primary stakeholder group, as the SD will be their primary way of getting things done at the UC and communicate with the UC's management.
  \item The UC's management will primarily communicate and negotiate with the students represented by the executive committee of the SD.
  \item Because of the nature of a UC, the UC's marketing department will benefit greatly from being able to boast a good and inclusive student community across faculties, which can easiest be achieved through a functional SD. This will, in turn, quite possibly affect prospective students to apply.
\end{itemize}

One key issue is that only the former NITH already has an active student community rooted in the SD. This means that the SD's budget will naturally be skewed towards the Faculty of Technology in the first period while a culture for collaborating across faculties is established. The only notable exception to this is a "party committee" comprising only a few students at the former Westerdals.

Another key issue is that neither existing governance takes into account the possibility of the SD in its entirety not being colocated. This introduces a whole class of practical concerns: for instance, no faculty is interested in local issues at other faculties such as missing mirrors or broken internet connections, which are common issues raised in the existing SDs. Further, it is impractical to represent all students at a single location more often than every couple of months, but a fast flow of communication regarding local issues is imperative to the function of the SD.

A final key issue is the student body's identification with the UCs they had applied to: there was already opposition to the merger and identifying with the UC's new name and profile, let alone feeling an affiliation with the other schools, which had previously been considered \textit{competition} by many.

Merging the SDs in a manner inclusive and efficient for everyone, and additionally speeding up processes that had already been slow where possible, will require a complete restructuring of the existing organisation into a new, single organisation with different branches. For example, communication between faculties will be a big issue not previously addressed by the existing SDs. This means:

\begin{itemize}
  \item redefining existing roles,
  \item introducing new roles, and
  \item removing existing roles that would impede the new SD from working optimally.
\end{itemize}

\citeauthor{luftman:managing-the-information-technology-resource:2009} (\citeyear[1232]{luftman:managing-the-information-technology-resource:2009}) states that:

\begin{quote}
    One of the best examples of how IT can change the way a business operates is found in the concept of \textit{disintermediation}.
\end{quote}

\citeauthor{luftman:managing-the-information-technology-resource:2009} specifically considers IT, but disintermediation (the removal a "middle man" in a process which is no longer required) is sure to be part of the change process, but only intended to empower the different positions. While this is sure to affect individuals, they must be provided with a way to clearly see that while their current positions are at stake, they will still be able to bring value to the new organisation in the same manner with altered roles and still bring great value to the new SD if only they adapt to the change.
