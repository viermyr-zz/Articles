\section{Literature review}

Having established the background for the change effort, it becomes apparent that the areas of \textit{change management} and \textit{merging organisations like the student democracy} with attention to handling opposition rooted in ethical grounds will be imperative to the success of the change effort.

\citeauthor{pettigrew-woodman-cameron:studying-organizational-change-and-development:2001} (\citeyear{pettigrew-woodman-cameron:studying-organizational-change-and-development:2001}) performs a thorough literature review of the 44th edition of the \textit{Academy of Management Journal}, pointing out challenges for future research regarding organisational change. In particular, it identifies that very few articles focused on linking scholarship and practise, linking process to outcome, and international comparative research. We will see that these areas have received some attention in later years, but that we still have a ways to go.

\citeauthor{o-mahoney-markham:management-consultancy:2013} (\citeyear{o-mahoney-markham:management-consultancy:2013}) thoroughly considers ethics of management consultancy and change efforts from a consultancy perspective, and raises several key issues concerning how a consultant should introduce and manage projects -- but does not consider change efforts initiated from within the organisation, although many parallels can be drawn.

\citeauthor{hargie:skilled-interpersonal-communication-research-theory-and-practice:2011} (\citeyear{hargie:skilled-interpersonal-communication-research-theory-and-practice:2011}) focuses communication, which is of course a key aspect of organisational transformation and handling of opposition, but not on the overarching processes. However, the contribution is indeed important to the field, and should be acknowledged.

\citeauthor{luftman:managing-the-information-technology-resource:2009} (\citeyear{luftman:managing-the-information-technology-resource:2009}) considers the importance of IT in introducing change in an organisation, and how the IT department must obtain a somewhat complete perspective of the organisation before committing to introducing change. On a related note, \autocite{joseph-ang-chang-slaughter:practical-intelligence-in-it:2010} considers the importance of soft skills in IT personnel, exactly due to the power \citeauthor{luftman:managing-the-information-technology-resource:2009} presents that IT has over organisational processes. \autocite{luftman:managing-the-information-technology-resource:2009} also considers general change management, and in particular, the absolutely imperative importance of communication.

\citeauthor{luftman:managing-the-information-technology-resource:2009} (\citeyear{luftman:managing-the-information-technology-resource:2009}) is backed up by \autocites{kotter:leading-change-why-transformation-efforts-fail:1995}, which considers eight common errors commonly made when introducing a change effort in an organisation.

\citeauthor{peng-liu-tao:analyzing-the-pathway-of-organizational-change-based-on-the-environmental-complexity} (\citeyear{peng-liu-tao:analyzing-the-pathway-of-organizational-change-based-on-the-environmental-complexity}) considers the motives behind, characteristics of, and goals of organisational change, and clearly states that organisational development is a dynamic and continuous process. This article concludes with several modes of change.

\citeauthor{lin-wei:the-role-of-business-ethics-in-merger-and-acquisition-success:2006} (\citeyear{lin-wei:the-role-of-business-ethics-in-merger-and-acquisition-success:2006}) explicitly considers the role of business ethics in merger and acquisition success in the \textit{Journal of Business Ethics}, providing an important contribution in this area.

Finally, \autocite{mcgee:some-overlooked-ethical-issues-in-acquisitions-and-mergers:2004} considers the other side, emphasising ethics of handling the resistance to the change with regard to several tactics used defensively against change efforts.
