\section{Conclusion and further research}

Somewhat in the manner suggested by \autocite{pettigrew-woodman-cameron:studying-organizational-change-and-development:2001}'s extensive review of the journal in which it was published, this article has attempted to deliver "how to" knowledge in the field of merging organisations. Yet, this is only one context: one key issue is that there are no paid positions in the student democracy, which certainly will have an effect on the change. Further, no one student is (normally) a part of the student democracy for more than three (or at an absolute high, five) years, which means that ownership will not be as present as in other organisations. This also likely means that a cultural change is far less difficult than it would have been in an organisation where the upper management is well defined and established, and there is already established a "way we do things around here".

The change will be forced upon the students whether they want to be a part of it or not, and the resistance to the organisational change will almost always be rooted in fear of being underrepresented or losing power, which is of course not the intention of the change effort. The best way by far to limit paranoia regarding the change effort is to expose the effort through a clearly defined and regularly updated vision, and be open about all parts of the effort in a way that continuously reaches all stakeholders.

\autocite{pettigrew-woodman-cameron:studying-organizational-change-and-development:2001} identified several key research areas, and these still seem to require attention. In addition, it seems that the ethics concerning failed mergers and acquisitions are still a gap in the current literature.
