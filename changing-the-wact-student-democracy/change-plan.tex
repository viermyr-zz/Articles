\section{Change plan}

Opposition rooted in both individuals's fear for their own positions, trouble with feeling any affiliation to the new SD in the short term, as well as ethical concerns from the student body at large and the representatives in the SD must therefore be expected. In particular, as found by \autocite{lin-wei:the-role-of-business-ethics-in-merger-and-acquisition-success:2006}, ethical areas that will need to be addressed explicitly include \textit{employment security} (even though "employment" is possibly the wrong term in this case), \textit{justice} in the form of an appraisal system, and \textit{caring practises} both during and after the initial change effort. Additionally, \autocite[8]{mcgee:some-overlooked-ethical-issues-in-acquisitions-and-mergers:2004} considers the possibility of the merger \textit{failing} -- this must certainly be considered in the process.

Using the eight steps to transforming an organisation form \citeauthor{kotter:leading-change-why-transformation-efforts-fail:1995}'s \citeyear{kotter:leading-change-why-transformation-efforts-fail:1995} article \citetitle{kotter:leading-change-why-transformation-efforts-fail:1995} as a basic framework, this chapter presents an action plan for conducting the transformation. This section, therefore, naturally leans heavily on the matter discussed in this article.

\subsection{Establishing a sense of urgency}

The sense of urgency is to some extent already established because of the merger, but the importance of creating a functional and flourishing student community may still be lost. The guiding coalition discussed in the next subsection will need to communicate this sense of urgency to both the student body as a whole, and the UC's management to foster productive collaboration.

This must be done by exposing problems with the status quo, which can for example be done as exemplified in \autocite{kotter:leading-change-why-transformation-efforts-fail:1995} by performing a satisfaction survey and making the findings public. This should encourage both the UC's management and the student body to take action.

As previously mentioned, a key issue with the status quo is the distribution of students across the faculties, and how the SD's budget is only focused towards one of the five faculties. In order to maintain a fair and inclusive student community and democracy, it is important to form a guiding coalition that can work towards these factors.

\subsection{Forming a powerful guiding coalition}

As seen in the previous subsection, a powerful guiding coalition with the means to act on the change initiative comprising "high-profile" students experienced with the existing student democracy, representing all faculties to as great an extent as possible, must be established if the interest of the student body is to be kept in focus as the change progresses.

The guiding (student) coalition must be in close contact with the UC's management to keep the process aligned with the UC's strategy. Further, this student coalition must, of course, communicate regularly with the students while it is working.

The most important work completed by the coalition will be to create and establish a vision, and empower others to act on the vision -- in addition to maintaining the sense of urgency and focusing on creating short-term wins with the long-term goal in mind to convert the opposition.

\subsection{Creating a vision}

Establishing a vision that has the student body's interests in mind while still working with the UC's management in a realistic fashion is imperative to the change effort's success.

An important factor to consider is that the vision must be continuously iterated upon as the change effort progresses, as opposed to leaving it as a static statement that will be ignored and forgotten; clearly, the guiding coalition will not have all the answers from the beginning of the effort.

As we know that some of the resistance towards the change effort will be rooted in ethics, it is important the effort has transparency in this area in mind from the very beginning, starting with the vision. This way, it will become possible to involve rather than alienate the opposing forces concerned about ethics in the newly formed organisation.

The vision will be the starting point for forming a strategy, which (if communicated transparently and well) will allow other parts of the organisation to act on the change in a productive manner. Failing to establish a clear and flexible vision (and in turn, strategy) will only let the opposition gather forces, as the end goal will be lost from sight \autocite{luftman:managing-the-information-technology-resource:2009}, \autocite{peng-liu-tao:analyzing-the-pathway-of-organizational-change-based-on-the-environmental-complexity}.

\subsection{Communicating the vision}

\citeauthor{kotter:leading-change-why-transformation-efforts-fail:1995} (\citeyear{kotter:leading-change-why-transformation-efforts-fail:1995}) establishes that communicating the vision clearly and regularly to all affected parts of the organisation is imperative to prevent the change opposition from gathering forces against something unknown and new which threatens their positions. The guiding coalition must therefore establish routines for communicating the vision to all students in all faculties, as well as to the UC's management.

Clear and regular communication of the vision will also be important in demonstrating short-term wins, as we will see later: when many know of the effort, it becomes easy to tie small victories to the long-term strategy.

The newly merged UC already suffers from having no way for the students to communicate across faculties, so the guiding coalition must create a way to transparently make the status of the efforts (as well as short-term wins) available to the students and other stakeholders. An effective strategy can be to build an interactive and actually useful website for the SD. This website should focus on not only providing static data, but make progress transparent. This changes the very core of the SD, as there will suddenly be an IT department to manage. This will introduce a whole class of new considerations as it changes the entire core of the organisation \autocite{luftman:managing-the-information-technology-resource:2009}, especially if the coalition chooses to use external workforce \autocite{o-mahoney-markham:management-consultancy:2013}. In turn, this system can be used to demonstrate short-term wins, as will be discussed later.

\subsection{Empowering others to act on the vision}

While the guiding coalition does its work establishing the new democracy and organisation, the SD must, of course, continue to exist. It is therefore paramount that others are empowered to act on the change effort while it progresses, to ensure that the opposition does not gather forces against something that cannot be used.

Members of former committees and groups from the separate SDs must be involved in the work and continue to maintain the existing SD as needed, and should at all costs not be alienated by the guiding coalition as this will likely lead to them joining the opposition \autocite{kotter:leading-change-why-transformation-efforts-fail:1995}. Instead, they should be made explicitly aware of the planned and completed short-term wins discussed in the next subsection.

A concrete measure the guiding coalition should take here is to make the by-laws defining the SD more flexible by reducing the time between meetings empowered to change the by-laws and other guiding documents from one to one half year. This will enable the SD to modify itself as it sees fit, and encourage a healthy communication between the guiding coalition and the student body.

\subsection{Planning for and creating short-term wins}

As mentioned previously in this section, it is important to plan for and create short-term wins to win the opposition over little by little \autocite{kotter:leading-change-why-transformation-efforts-fail:1995}. Once presented with facts about how the change positively impacts the SD, the opposition should be convertable to instead join the change effort. This will be the first important step towards converting key, high profile members of the opposition to change advocates, or \textit{champions} \autocite{luftman:managing-the-information-technology-resource:2009}.

\citeauthor{kotter:leading-change-why-transformation-efforts-fail:1995} notes that showing results within 12--24 months is imperative. It follows, then, that the change coalition cannot spend too long planning the merger of the SDs: it must be put into action piece by piece and demonstrate that it is more efficient than the status quo.

Important short-term wins in this context may be that issues regarding single faculties are dealt with quicker than before while still being communicated well into the SD spanning all faculties, or that issues raised in meetings will be more relevant to the participants (and therefore shorter): students from one campus are not interested in broken mirrors or weak internet connections at other faculties, but need to be involved with the status quo. A final example is that the executive committee, which represents the SD between meetings, will be able to represent all faculties in a better fashion.

Planning for, creating, and demonstrating these short-term wins will boost the credibility of the change, while keeping the end goal of the change effort in sight \autocite{kotter:leading-change-why-transformation-efforts-fail:1995}.

\subsection{Consolidating improvements and producing still more change}

While focusing on short-term wins and consolidating their positive effects on the SD, more change must be produced continuously. Empowering the SD to change itself as it needs to, while simultaneously restricting it from going back to the status quo, will be of high importance. This can be done through utilising the change "champions" who advocate the change effort within the organisation.

Again, empowering biannual meetings to change the SD's by-laws and leading documents is a concrete measure that must be taken by the change coalition, as this is the only way to allow the SD to take change into its own hands and become an organisation of change champions rather than resistance \autocite[12395]{luftman:managing-the-information-technology-resource:2009}.

\subsection{Institutionalising new approaches}

\begin{quote}
    [\dots] change sticks when it becomes "the way we do things around here", when it seeps into the bloodstream of the corporate body.
\end{quote}
\autocite[9]{kotter:leading-change-why-transformation-efforts-fail:1995}

While the change coalition will be in the fortunate position of being able to "force" the democracy into working as the coalition intends from the beginning due to by-laws and routines effective from the very beginning of the actual change effort, resistance must naturally still be handled.

At the final stages of the initial change effort towards a merged and incluside SD, techniques like negotiation, coercion, and even manipulation or cooptation may be required to convert the remaining resistance to join the effort.
